% Stefan Kottwitz, LaTeX Cookbook, Packt Publishing, 2015
% Chapter 5, Designing Tables
% Adding footnotes to a table
\documentclass{article}
\usepackage{booktabs}
\usepackage{bbding}
\usepackage{threeparttable}
\begin{document}
\begin{table}
  \centering
  \renewcommand{\arraystretch}{1.6}
  \begin{threeparttable}
    \begin{tabular}{lccccc}
      \toprule
      Class   & Part page  & Chapters   & Abstract\tnote{1} &
      Front-/Backmatter\tnote{2}  & Appendix name\tnote{3} \\
      \cmidrule(r){1-1}\cmidrule(lr){2-2}\cmidrule(lr){3-3}
      \cmidrule(lr){4-4}\cmidrule(lr){5-5}\cmidrule(l){6-6}
      article &            &            & \Checkmark &  \\
      book    & \Checkmark & \Checkmark &            &
                \Checkmark & \Checkmark                 \\
      report  & \Checkmark & \Checkmark & \Checkmark &
                           & \Checkmark                 \\
      \bottomrule
    \end{tabular}
    \begin{tablenotes}
      \item[1] An environment: \verb|\begin{abstract}| \ldots
               \verb|\end{abstract}|
      \item[2] Commands: \verb|\frontmatter|, \verb|\mainmatter|,
        \verb|\backmatter|
      \item[3] The \verb|article| class provides the
        \verb|\appendix| command without
        ``Appendix'' prefix.
    \end{tablenotes}
  \end{threeparttable}
  \caption{Structuring differences between standard
           \LaTeX\ classes}
  \label{comparison}
\end{table}
\end{document}